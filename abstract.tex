\setcounter{chapter}{0}
\setcounter{secnumdepth}{0}
\cfoot{\thepage}
%\pagenumbering{roman}
%\pagenumbering{gobble}
\setcounter{page}{v}
%\pagenumbering{roman}

\sectionfont{\centering\MakeUppercase}

%\pagenumbering{roman}
\chapter*{ABSTRACT}
\addcontentsline{toc}{chapter}{ABSTRACT}

Typhoid fever is a major cause of morbidity and mortality in low and middle-income countries. A recent epidemic of typhoid fever in Blantyre saw cases rise from 67 in 2011 to 782 in 2014. The morbidity, carriage and genomic epidemiology of typhoid (MCET) study which was conducted by Malawi-Liverpol Wellcome Trust at Queen Elizabeth Central Hospital between 2015 and 2016 found that 7 genomic sub-lineages of H58 lineage of \textit{S. typhi} were causing the outbreak. The spatial and spatio-temporal distribution of the 7 sub-lineages, which may help to explain their transmission routes, has not yet been described. Log-Gaussian Cox Process (LGCP) models were used for the spatial and spatio-temporal analyses. Specifically, a multi-type spatial LGCP model was used for an overall spatial analysis and four spatio-temporal LGCP models (all cases and stratified by sub-lineage) were fitted for a spatio-temporal analysis of the data. The parameters in the LGCP models include $\sigma$ with median of 2.185 (95\% Credible Interval (CrI) 1.93 to 2.497); the parameter $\phi$ had a median of 940.1 metres (95\% CrI 709 to 1275); and the parameter $\theta$ had a median of 0.075 months (95\% CrI 0.050 to 0.107). $\sigma$ is the standard deviation parameter which scales the log-intensity, whilst the parameters $\phi$ and $\theta$ govern the rates at which the correlation function decreases in space and in time respectively. The long term distribution of the typhoid cases show that the outbreak was at its peak between October and November 2015. The analysis provides evidence for sub-lineage specific spatial distribution of the H58 lineage of \textit{S. typhi} during the recent typhoid outbreak in Blantyre, Malawi.

\newpage