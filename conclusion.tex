\newgeometry{left=1.5in, right=1in, top=2in, bottom=1in}

\chapter{CONCLUSION, RECOMMENDATIONS, LIMITATIONS AND AREA FOR FURTHER RESEARCH}

\section{Conclusion}

The typhoid fever cases which were recorded at QECH between March 2015 and December 2016 were all caused by seven different sub-lineages of the H58 lineage of \textit{S. typhi}. The long term distribution of the typhoid cases shows that since the start of the study (March 2015), the outbreak was increasing steadily until around October and November 2015 when the outbreak was at its peak with a range of 15 to 25 cases per month. Then the cases started dropping until July 2016 when the cases started increasing again.

The spatio-temporal models have also shown that cases of clade 0 and clade 2 sub-lineages had similar temporal trend to the long term trend of all the typhoid cases combined. However, the temporal distributions cannot be used to assess seasonality because of limited time points. The maximum time points used in the study was 22 points i.e. 22 months. The minimum being 15 points.

The multi-type spatial model has shown that clade 0, clade 2 and the grouped clades all had their own high-transmission locations distinct from each other with minor overlaps. This proves that the clades were competing against each other and local transmission were happening around existing cases or potentially specific water sources becoming contaminated with specific H58 \textit{S. typhi} sub-lineage. Typhoid fever cases caused by clade 0 sub-lineage were dominant in the western and south eastern side of Blantyre city. The areas include Chirimba, Kameza, Machinjiri, 

\restoregeometry

Nkolokoti-Kachere, Likhubula, Mbayani-Chemusa, Nancholi-Manase and Bangwe-Namiyango. clade 2 cases were dominant in Ndirande and Chilomoni. The grouped clades were dominant in the eastern side of the city. The areas include Nancholi, Zingwangwa, Cholobwe-Misesa, Chigumula, Zingwangwa, Kachere, Mapanga, Machinjiri and Kameza.

\subsection{Recommendation}

The researcher has made the following recommendations based on the spatial and spatio-temporal LGCP models implemented in this research:

\begin{enumerate}[i.]
\item The analysis should be done using data which have more time points. This will help to assess seasonality effects.
\item Include environmental and economic factors like elevation, temperature and closeness to water sources in the model to also assess the effects of these factors and how they affect the spatial and temporal distribution of typhoid fever in Blantyre city.
\item The Ministry of Health (MoH) and the Ministry of Water and Sanitation (MoWS) should work together to enhance water and sanitation services delivery in the areas with high incidence rates of typhoid fever.
\end{enumerate}

\subsection{Limitations}

The study failed to assess seasonality as a temporal covariate. This is because the MCET dataset only had 22 time points. The study also failed to fit models for all the 7 sub-lineages because some sub-lineages had very few cases for proper model fitting. That is why other sub-lineages with few cases were grouped into a single sub-lineage. 

The thesis also failed to fit a multi-type spatio-temporal LGCP model because of the current limitation of the \textit{lgcp} R package. Further, the \textit{lgcp} package only fits LGCP models which assumes a separable covariance function for the Gaussian process.

\subsection{Areas for Further Research}

For further studies, there is need to fit a multi-variate spatio-temporal LGCP model where spatial and temporal relationships and interactions among sub-lineages can be investigated. This was currently not done because it was beyond the scope of this thesis.

Further studies should also incorporate environmental and economic factors apart from just assessing spatial and temporal factors. The effects of these environmental factors would help to shade more light why some areas had high incidence rate of typhoid fever than the other as found in this study. This was also not done because it was beyond the scope of this thesis.
