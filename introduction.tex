\newgeometry{left=1.5in, right=1in, top=2in, bottom=1in}

\chapter{INTRODUCTION}

\section{Background Information}

Bacteria of the genus \textit{Salmonella} are a major cause of foodborne illness throughout the world. As a zoonotic pathogen, salmonella can be found in the intestines of many food-producing animals such as poultry and pigs. Infection is usually acquired by consumption of contaminated water or food of animal origin: mainly undercooked meat, poultry, eggs and milk. Human or animal faeces can also contaminate the surface of fruits and vegetables, which can lead to foodborne outbreaks \citep{Who:2014}.

Most salmonella strains cause gastroenteritis, while some strains, particularly \textit{Salmonella enterica} serotypes Typhi and Paratyphi, are more invasive and typically cause enteric fever. Enteric (typhoid) fever is a more serious infection that poses problems for treatment due to Antibacterial Resistance (ABR) in many parts of the world. According to the study on the burden of typhoid fever in low and middle income countries suggests that 17.8 million (95\% CrI 6.9 to 48.4 million) cases occur annually \citep{Antillon:2017}. However, other studies have suggest that the statistics for Sub-Saharan Africa (SSA) are not accurate because of limited health facilities with microbiological diagnostic capabilities \citep{Peters:2004}. As a result, others have suggested that the burden of typhoid fever in Africa may be over-estimated \citep{Mweu:2008}.

\restoregeometry

\subsection{Salmonella Typhi in Blantyre}

A longitudinal health surveillance study done in Blantyre, Malawi has shown that before 2010, most of the bloodstream infections (BSI) registered at Queen Elizabeth Central Hospital (QECH) were caused by multidrug resistant (MDR) nontyphoidal serovars of \textit{Salmonella} (NTS) while \textit{S. typhi} only caused 1\% of the BSI \citep{Gordon:2008, Musicha:2017, Feasey:2015}. The study found out that between 1998 and 2010, there were only 176 microbiologically confirmed cases of \textit{S. typhi} at QECH in Blantyre. This represents an average of 14 cases per year. Only 12 of the 176 cases were found to be MDR to ampicillin, chloramphenicol and cotrimoxazole \citep{Feasey:2015}. However, from 2011, the surveillance study showed a rapid increase in microbiologically confirmed \textit{S. typhi}. For example, 67 typhoid fever cases were confirmed in 2011 followed by 186 cases in 2012 and 843 cases in 2013 and 782 cases in 2014 \citep{Feasey:2015}.

In trying to understand the transmission routes of the rapid increase in cases of microbiologically confirmed \textit{S. typhi} infections, several studies were conducted. One of the studies was the morbidity, carriage and genomic epidemiology of typhoid (MCET) \citep{Feasey:2015}. The aim of the study was to investigate whether the typhoid fever cases were caused by a single lineage of \textit{S. typhi}, and to describe the full diversity of \textit{S. typhi} in Blantyre.

In the study, patients under the age of 10 diagnosed with culture-confirmed typhoid fever at QECH in Blantyre were recruited in the prospective observational cohort study \citep{Feasey:2015}. Controls were recruited in the ratio of 4 to 1 \citep{Gauld:2020}. A total of 314 cases consented to provide their household locations, and 256 isolates were whole genome sequenced. The results showed that prior to 2011, typhoid fever cases were being caused by four different \textit{S. typhi} haplotype/lineages (H42,H52,H50 and H55). Typhoid fever cases caused by H58 lineage "rapidly expanded in 2011" \citep{Feasey:2015}. By 2013, all typhoid fever cases which were being registered at QECH were caused by H58-haplotype \citep{Feasey:2015}. Further analysis of the MCET data revealed that there are 7 sub-lineages of the H58 lineage which were causing the typhoid fever outbreak \citep{Wailan:2019}.

\subsection{Factors Associated with Typhoid Fever}

Further analysis of the MCET data provided detailed insight into the risk factors for paediatric typhoid fever in Blantyre. The findings point to complex and varied risk factors including water source, household sanitation and hygiene, and social interaction patterns such as school attendance \citep{Gauld:2020}. Cooking and cleaning with water from an open dug well was also identified as a risk factor. Sources of drinking water were found not to be associated with typhoid. Potential explanation was that communities are aware of the risks associated with drinking unclean water, but less aware of the risks of indirect exposure, such as through pans or other items that may come into contact with food.

Another explanation was that people may prioritize safe water for drinking but cannot afford to purchase or transport the volume of safe water needed for use in other household tasks. It is estimated that less than 5\% of the Blantyre city population is connected to the sewage network, with the majority of the population utilizing pit latrines \citep{Gauld:2020}. During rainy season, the runoff from the pit latrines may contaminate unprotected wells and rivers. The water from these sources are used for cooking. The study also found out that the risk of typhoid increases when using multiple drinking water sources \citep{Gauld:2020}.

\subsection{Spatial-Genomic Analysis of Salmonella Typhi in Blantyre}

Spatial-genomic data analysis of the MCET study data was done to find out if it can help shed more light on the transmission routes of the disease \citep{Gauld:2020}. In the spatial-genomic analysis, a Poisson log-linear model was used to model typhoid incidences across the city, initially with the assumption of no spatial dependence. Covariates used in the model include distance to QECH, elevation and river catchment at the centroid of the Enumeration Area (EA), and average household size and population density per square km across the enumeration area.

The analysis revealed a heterogenous distribution of \textit{S. typhi} isolates across the city \citep{Gauld:2022}. The practical range of spatial correlation was approximately 192 meters, indicating the model’s spatial random effect was capturing short-distance spatial correlation. Although the city$'$s geographical range spans approximately 20 kilometres, households in the cohort are clustered. A significant correlation between spatial and genetic distance was subsequently found, showing typhoid fever patients living closer together were more likely to have S. Typhi isolates with closely related genomes. The analysis also showed that elevation, as a spatial factor, was not a significant risk factor of typhoid fever. This is contrary to other studies which found that low-laying areas are associated with high risk of typhoid fever \citep{Akullian:2015}.

\subsection{Point Pattern Analysis}

Point pattern analysis focuses on describing patterns of points over space and time, and making inference about the process that could have generated an observed pattern. The main focus lies on the information carried in the locations of the points, and typically these locations are not controlled by sampling but as a result of a process of interest like typhoid fever case. Point pattern analysis is different from geostatistics where the main interest is not in the observation locations but in estimating the value of the observed phenomenon at unobserved locations. Point pattern analysis typically assumes that for an observed area, all points are available, meaning that locations without a point are not unobserved as in a geostatistics, but are observed and contain no phenomenon of interest. In point processes, locations are treated as random variables, whereas in geostatistics, the measured variable is a random variable on fixed locations.

\section{Problem Statement}

Sometimes, descriptive statistics may not be enough to fully understand the dynamics of the point process up until a spatio-temporal model is developed. Since MCET data is point pattern, spatio-temporal point process framework is better suited to estimate the intensity function which predicts the rate of typhoid fever events in space and time.

The simplest case of these class of models is the homogeneous Poisson process where the intensity is constant in space and time. A more flexible inhomogeneous model is the log-Gaussian Cox process in which the log intensity is assumed to be drawn from a Gaussian process \citep{Diggle:2013}. With a suitable choice of spatio-temporal correlation function, the underlying Gaussian process can be estimated. It is also statistically prudent to model both the population density and risk as continuous phenomena in time and space while recognising, firstly that the available data will be spatially incomplete and/or aggregated as well as susceptible to measurement error, and secondly that even after modelling the effects of all candidate variables, there will often be a residual component of spatio-temporal variation in risk that can only be captured by including in the model one or more latent, spatio-temporal stochastic processes.

Other classes of spatio-temporal models have been used to analyse Salmonella infections in farm animals like dairy cattle \citep{Fenton:2009, Cox:2012}. In these studies, Markov Chain model with transition probabilities and generalized linear spatial model were used to estimate the spatial and temporal patterns of the Salmonella infections in dairy herds. Ripley’s K function was used to statistically identify disease clusters. The limitation of these modelling framework is that they do not incorporate the presence of a population at risk and a combination of environmental and individual characteristics that affect the risk of disease at each location in space and time in the model. Spatio-temporal point pattern process, which will be used in this project, was used to analyse the spread of Avian Influenza Virus (H5N1) in Turkey. The intensity function which was used was based on a self exciting point process framework which has successfully been used for modelling earthquakes \citep{Kim:2011, Ogata:1998}.

All the previous analyses on S. Typhi in Blantyre, Malawi did not look at the spatio-temporal point pattern signals of the sub-lineages of H58 of S.Typhi. This project, which will use the MCET dataset to fit spatial and spatio-temporal point pattern statistical models of the 7 sub-lineages of H58 lineage of S. Typhi using a log-Gaussian Cox Process as its modelling framework.

\section{Hypothesis of the Study}

The hypothesis of the study is that different sub-lineages have different spatio-temporal dynamics.

\section{Research Questions}

Based on the exploratory data analysis and the findings from the MCET study, this project will be guided by the following research questions:

\begin{enumerate}[i.]
    \item What is the spatio-temporal distribution for typhoid cases in Blantyre?
    \item Are there differences in spatial and temporal distributions between sub-lineages?
    \item Is there evidence for interactions (competition, synergy) between sub-lineages?
\end{enumerate}

\section{Objectives of the Study}

The overall objective of this thesis project is to conduct spatial and spatio-temporal analysis of the 7 sub-lineages of H58 \textit{S. typhi} lineage during the 2015-16 typhoid epidemic in Blantyre, Malawi.

\subsection{Specific Objectives}

\begin{enumerate}[i.]
    \item To describe the spatial variations of the distribution of typhoid cases by sub-lineage.
    \item To describe the temporal variations of the distribution of typhoid cases by sub-lineage.
    \item To investigate the existence of spatial and/or temporal interactions of the 7 sub-lineages among typhoid cases.
\end{enumerate}

\section{Significance of the Study}

Public health interventions for typhoid are challenging because typhoid fever incidences are often dynamic in space and time, and transmission routes are not consistent across locations and time. The results of the spatio-temporal point process model will, in general, add new knowledge to the existing body of knowledge about typhoid fever and specifically explain further the transmission routes of the H58 lineage in Blantyre, Malawi. The results would also help to develop targeted public health intervention to prevent the re-emergence of the outbreak in future.

The Ministry of Health (MoH) can also benefit from the spatio-temporal modelling framework if the modelling framework can be incorporated into the real-time surveillance systems in government hospitals and clinics. The analysis can help to identify spatio-temporal clusters of infectious diseases in real-time. The MoH can use its limited resources efficiently by prioritizing assistance to where and when it is needed most in the fight to control future infectious disease outbreaks.
