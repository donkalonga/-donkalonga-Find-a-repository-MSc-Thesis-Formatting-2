\newgeometry{left=1.5in, right=1in, top=2in, bottom=1in}

\chapter{LITERATURE REVIEW}

\section{Introduction}

The chapter will review some of the main concepts which are used in spatial and spatio-temporal modelling. The chapter will also review the available literature on spatio-temporal modelling including point process spatio-temporal modelling of infectious diseases.

\section{Types of Spatial Data}

According to \citet{Cressie:2015}, spatial data are classified into three basic types: geostatistical data, spatial point pattern data and areal spatial data. \textit{Geostatistical data} are point observations of a continuous varying quantity over a spatial region. \textit{Spatial Point Pattern Data} are when $Y(s)$ is a random vector at a location $s \in \mathbb{R}^d$ where $s$ varies consistently over $D$, a fixed subset of $\mathbb{R}^d$ that contains a $d-$dimensional rectangle of positive volume. The data are in the form of points or events irregularly distributed within a region of space. \textit{Areal Spatial Data} are when $D$ is a fixed subset (of either regular or irregular shape) but partitioned into a finite number of areal units with well-defined boundaries. Areal data are also called lattice data or aggregated data. Aggregated data usually consist of data which have been summarised into means of areal units which contain individuals that are close together \citep{Cressie:2015}. According to \citet{Tranmer:1998}, data aggregation process leads to loss of information about individual location and variation within groups. This means that the analysis of the aggregated data cannot be used to infer individual relationships. Since the boundaries of the groups are imposed and not 

\restoregeometry


natural, results of the analysis can change when boundaries change. Mathematically, lattice data can be defined as $\textbf{Z} \equiv {Z(s_1),...,Z(s_m)}$, where ${s_1,...,s_m}$ are reference locations for data defined on discrete spatial features \citep{Cressie:2015}.  The rest of the thesis focuses on spatial point pattern data.

\section{Spatial and Spatio-Temporal Modelling Concepts}

The section will present some of the key concepts used in spatial and spatio-temporal point process modelling.

\subsection{Stochastic Process} \label{2.3.1}
 
A stochastic process is a sequence of random variables $\lbrace X(t):t\in T \rbrace$, where $t$ denotes observed time and $T$ is the sample space which can be either discrete or continuous and can contain both negative and positive values. The stochastic process $X(t)$ is said to be strictly stationary if the distribution of $X(t_1),...,X(t_n)$ is the same as that of  $X(t_1 + \tau),...,X(t_n + \tau)$ for all choices of time lag $\tau$ and time points $t_1,...,t_n$. Specifically, writing $p(.)$ for the distribution function, $\lbrace X(t)\rbrace$ is said to be strictly stationary if $p(X(t_1),...,X(t_n)) = p(X(t_1 + \tau),...,X(t_n + \tau))$ for all sets of time points $t_1,...,t_n$ and all time lags $\tau$.

The basic idea of process modelling is to construct a model of a process starting from a set of sequences of events assumed to have been generated by the process itself. Subsequently, the model could be also used to discover properties of the process, or to predict future events on the basis of the past history. From a general point of view, a process model can be used for three main purposes: describing the details of a process; predicting its outcomes; predicting the effects of independent variables which have been incorporated in the model. This is different from a deterministic model, which predicts outcomes with certainty, with a set of equations that describe the system inputs and outputs exactly. Therefore, a stochastic model represents a situation where uncertainty is present. In other words, it is a model for a process that has some kind of randomness.

Gaussian processes are stochastic processes that are governed by their first two moments. As such, modelling a Gaussian process involves specifying the mean and the covariance structure. A second-order stationary spatio-temporal process ${\eta(s,t):(s,t))\in \mathbb{R}^d \times \mathbb{R}}$ has a constant first moment and there exists a function $C$ defined on $\mathbb{R}^d \times \mathbb{R}$ such that $Cov \lbrace \eta(s+h,t+u), \eta(s,t)\rbrace = C(h,u)$ for $s,h \in \mathbb{R}^d$ and $t,u \in \mathbb{R}$. The function $C$ is the space-time covariance function of the Gaussian process and its margins, $C(.,0)$ and $C(0,.)$, are purely spatial and purely temporal covariance functions respectively \citep{Gelfand:2010}. "A space-time covariance function is separable if there exists purely spatial and purely temporal covariance functions $C_s$ and $C_t$ such that $C(h,u) = C_s(h) . C_t(u)$ for all $(h,u \in \mathbb{R}^d \times \mathbb{R})$" \citep{Gelfand:2010}. Non-separable stationary covariance functions are more realistic in real life modelling because they accommodates space-time interactions.

\subsection{Spatial and Spatio-Temporal Point Process} \label{2.3.2}

\citet{Diggle:2013} defines a spatial point process as a "stochastic mechanism that generates a countable set of events $s_i$ in the plane". A spatial point process can also be defined as a stochastic process governing the location of events ${s_i}$ in some set $D_s \subset \mathbb{R}^2$, where the number of such events in $D_s$ is also random \citep{Cressie:2015}. Mathematically, a spatio-temporal point process is defined as a mechanism which generates a countable set of events $D_{s,t}$ in  $\mathbb{R}^2 * \mathbb{R^+}$. Suppose $T$ denotes the largest time in the sample space; then there exists a set of spatio-temporal point process realisations $D_{s,t}$ in  $\mathbb{R}^2 * \mathbb{R^+}$ such that $D_{s,t} \subset D_s * [0,T]$ \citep{Cressie:2015}. An intensity function of a spatio-temporal point process, denoted as $\lambda (s,t)$, is the mean number of events per unit area and time. The intensity of a point process is fundamental in understanding the pattern of the realisation of the point process.

\subsection{Marked Spatio-Temporal Point Process} \label{2.3.3}

Spatio-temporal point process may be marked if features of events beyond their time and location are also observed. Mathematically, \citet{Reinhart:2018}, presented the marked spatio-temporal point process  as a point process of event ${(s_i,t_i,k_i)}$, where $s_i \in X \subseteq \mathbb{R}^d, t_i \in [0,T)$, and $k_i \in K$, where $K$ is the mark space. A special case is the multivariate point process in which the mark space is a finite set ${1,...,m}$ for a finite integer $m$. Often the mark in a multivariate point process indicates the type of each event, such as the sub-lineages of H58 lineage of \textit{S. typhi}. A point process has independent marks if given the locations and times ${(s_i,t_i)}$ of events, the marks are mutually independent of each other and the distribution of $k_i$ depends only on ${s_i,t_i}$ \citep{Reinhart:2018}.

\subsection{Poisson Process}

The Poisson distribution is a discrete distribution that measures the probability of a given number of events happening in a specified time period. On the other hand, a Poisson process is a series of discrete event where the average time between events is known, but the exact timing of events is random. This shows that Poisson processes are associated with Poisson distribution. A spatial point process is said to be a homogeneous Poisson process when its intensity, $\lambda$, is constant across the a bounded region $A$. In most cases, a homogeneous Poisson model is used as a null model against which spatial point patterns are compared. On the other hand, an inhomogeneous Poisson process assumes a nonconstant intensity function within a bounded region.

\subsection{Log-Gaussian Cox Process (LGCP)}

A Cox process is defined as a "doubly stochastic" process  because it is an inhomogeneous Poisson process with a random intensity function. A spatio-temporal Cox process is a spatio-temporal Poisson process whose intensity is a realization of a spatio-temporal stochastic process $\Lambda(s,t)$. Log-Gaussian Cox process (LGCP) is the Cox process where the log intensity function is the Gaussian process. In their seminal paper, \citet{Moller:1998} demonstrated the remarkable ease with which we may theoretically decompose the log-Gaussian Cox process (LGCP), and the impressive flexibility possessed by this process with respect to capturing a wide variety of spatial intensity functions on $\mathbb{R}$. This makes the LGCP better suited for solving problems in, for example, geographical epidemiology.  \citet{Diggle:2001} later extended the application of the LGCP to the spatio-temporal setting.

Below is the general definition of the intensity function for the spatio-temporal log-Gaussian Cox process models.

\begin{align} \label{eqn1}
X(s,t) &= Pois \lbrace R(s,t)\rbrace \nonumber \\
R(s,t) &= C_A \lambda(s,t) exp \lbrace Z(s,t)\beta + Y(s,t)\rbrace \quad \quad
\end{align}

From the model, $X(s,t)$ is the number of events in the cell of the computational grid containing the point $s$ at time $t$. $R(s,t)$ is the Poisson rate while $C_A$ is the cell area. $\lambda(s,t)$ is the population offset and $Z(s,t)$ is a vector of measured covariates. $Y(s,t)$ is the latent Gaussian process on the computational grid.  $\beta$ represents the covariate effects. Other model parameters to be estimated include $\eta = \lbrace log(\sigma),log(\phi),log(\theta) \rbrace$, the parameters controlling the assumed dependence structure of the spatio-temporal Gaussian process $Y(s,t)$. The standard deviation parameter $\sigma$ scales the log-intensity, whilst the parameters $\phi$ and $\theta$ govern the rates at which the correlation function decreases in space and in time respectively \citep{Diggle:2013}. Below is the intensity function of the multi-type spatial log-Gaussian Cox process model.

\begin{align} \label{eqn2}
X_k(s) &= Pois \lbrace R_k(s)\rbrace \nonumber\\
R_k(s) &= C_A \lambda(s) exp \lbrace Z_k(s)\beta_k + Y_k(s) + Y_{K+1}(s)\rbrace \quad k \in 1,...,K \quad  with \quad  K\geq 2
\end{align}


In this model, $X_k(s)$ represents the number of events of type $k$ in the computational grid cell containing the point $s$. $R_k(s)$ is the Poisson rate and $C_A$ is the cell area. $\lambda(s)$ is the population offset and $Z_k(s)$ is a vector of measured covariates and $Y_k(s)$ are latent Gaussian processes on the computational grid specific to a particular type. $Y_{K+1}$ is the latent Gaussian process which captures spatial variation common to all types. The other parameters in the model include $\beta_k$, the covariate effects for the $k$th type and $\eta_k = \lbrace log(\sigma_k),log(\phi_k)\rbrace$ the parameters of the process $Y_k$ for $k = 1,...,K+1$ on a log scale.  $\eta_k$ controls the assumed dependence structure of the spatial Gaussian process $Y(s)$. The parameter $\sigma_k$ scales the log-intensity and $\phi_k$ controls the rates at which the correlation function decreases in space for a particular type \citep{Taylor:2015}.

Since cases of diseases occur in a spatio-temporal continuum, location and time of registration of cases are affected by the presence of a population at risk, environmental factors and individual characteristics at that location and time. As such, LGCP are an important class of models for spatial and spatio-temporal point-pattern and lattice data as defined in sections \ref{2.3.2} and \ref{2.3.3} because they incorporate population at risk, environmental factors and individual characteristics at that location and time \citep{Taylor:2015}. 

LGCP models are also the best there is for point pattern because they recognize that the available data may be spatially incomplete and/or aggregated as well as susceptible to measurement error. They also recognize that even after modelling the effects of all environmental variables, there will often be a
residual component of spatio-temporal variation in risk that can only be captured by including in the model one or more latent spatio-temporal stochastic processes \citep{Taylor:2015}.

\section{Spatial and Spatio-Temporal Modelling Approaches in Real World}

This section discusses past statistical approaches to spatial and spatio-temporal modelling of typhoid fever or other infectious diseases.

\subsection{Spatial Poisson Log-linear Model}

\citet{Gauld:2022} analysed the MCET data which is also be used for the spatio-temporal analysis in this paper. Their aim was to estimate the incidence of typhoid fever across Blantyre city. They fitted a Poisson log-linear model using areal spatial data aggregated at EA level unlike point pattern data which will be used in this paper. The model included covariates like distance to QECH, elevation, river catchment at the centroid of the EA, average household size and population density per square km across the enumeration area. The analysis found that the typhoid fever cases where heterogeneously distributed across Blantyre city. The analysis also showed that elevation, a spatial factor, was not a significant risk factor of typhoid fever contrary to \citep{Akullian:2015} who found that low-laying areas have higher incidence of typhoid fever that highlands \citep{Gauld:2022, Akullian:2015}. The initial model assumption of no spatial dependence is one of the main weakness of the model. The model also failed to include an autoregressive term to handle spatial dependence among the typhoid cases. The model also neglected the temporal analysis of the disease outbreak.

\subsection{Epidemic Avian Influenza (EAI) Model}

\citet{Kim:2011} adapted a self exciting point process introduced by \citet{Hawkes:1974} to implement a point process spatio-temporal model to describe the spatial distribution of Turkey's first avian influenza in 2006. The intensity function of the self-exciting point process conditional to the past history of time and space of the Epidemic Avian Influenza (EAI) model was defined as follows:
\begin{align}
\lambda(x,y,t \mid H_t) &= \dfrac{E\left[ N(dt dx dy) \mid H_t\right]}{dt dx sy}\\
&= \lambda_B(x,y,t) + \sum_{i:t_i<t}\alpha f(x - x_i, y-y_i) g(t - t_i) h_{traff}(x,y)k(T(t_i))
\end{align}

where $i$ is an index for each of the 221 H5N1 outbreaks occurred in Turkey during 182 days between October 1, 2005 to March 31, 2006. $(x_i, y_i, t_i)$ represents the location and time of an outbreak $i$ and $H_i = {x_i, y_i, t_i ; t_i < t}$ represents the past history of outbreaks up to time $t$. The subscripts $B$ and $T$ of the conditional intensity represents background and triggering respectively.

The background conditional intensity was defined as $\bf(\lambda_B(x,y,t)) = ae^{-bR_{city}(x,y)} e^{-kT(t)}$ representing the background intensity and has two components one for space and the other for temporal patterns of the outbreaks. Backfitting, Expectation Maximization (EM) and Poorman's EM methods were used for parameter estimation of the EAI model \citep{Kim:2011}.

\subsection{Spatio-Temporal Interaction Effects Model for Zika Virus Disease (ZVD) and Dengue Fever}

\citet{MartinezBello:2018} estimated the parallel relative risk of Zika virus disease (ZVD) and dengue fever using spatio-temporal interaction effects models for one department and one city of Colombia during the 2015-2016 ZVD outbreak. The model used lattice data and was fitted using the integrated nested Laplace approximation (INLA) for parameter estimation \citep{MartinezBello:2018}. Even though the model incorporated inseparable spatio-temporal interactions, it modeled the ZVD and dengue separately before combining the risk to obtain a joint risk estimation.

\subsection{Log-Gaussian Cox Process Model for Ambulance Calls in Northern Sweden}

\citet{Bayisa:2020} analysed the spatio-temporal pattern of ambulance calls which occured between 2014 and 2018 in Sweden. An LGCP was used to model the ambulance calls. Spatial component of the stochastic intensity function was estimated using K-means clustering based bandwidth selection method. The temporal intensity component of the stochastic intensity function was estimated by means of Poisson regression model which had temporal covariates like days of a week and season of the year. Inhomogeneous $K-$function was used to assess the existance of spatio-temporal clustering. 
Minimum contrast technique was employed to estimate variance parameter, spatial correlation function and the temporal correlation function \citep{Bayisa:2020}. The analysis showed that LGCP was a suitable model for point pattern data.

\subsection{Multivariate log-Gaussian Cox Process Model for Modelling Bovine Tuberculosis (BTB)}

\citep{Diggle:2013} modelled BTB, an infectious disease in United Kingdom (UK), using multivariate LGCP model. As part of a national control strategy for the disease, herds in UK undergo regular inspection. In a study done in 2013, whole genome sequencing was done on the tuberculosis bacterium that caused the outbreak. It was found out that there were multiple strains which were causing the outbreak. This necessitated the fitting of the multi-type LGCP model given by \ref{eqn2}.

The model allows decomposition of the spatial variation in events of multiple types into variation associated with a particular type of event and variation common to all types. Thus, although each point type may display an individual spatial pattern, the process $Y_{K+1}$ captures area of high or low intensity that are common to all types.