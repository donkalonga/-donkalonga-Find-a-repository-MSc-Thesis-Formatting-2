\newgeometry{left=1.5in, right=1in, top=2in, bottom=1in}

\chapter{METHODOLOGY}

This chapter discusses the methodology used in this study including the study design, data collection and model fitting.

\section{Study Design}

The data used in this project is from the morbidity, carriage and genomic epidemiology of typhoid (MCET) study which was conducted between March 2015 and December 2016 at QECH. The hospital provides free secondary healthcare to the Blantyre urban area and surrounding district and tertiary care to the southern region of Malawi. Since 1998, the Malawi Liverpool Wellcome Programme conducted sentinel surveillance of BSI at QECH \citep{Musicha:2017}. Patients under 10 years living in Blantyre who had blood culture confirmed typhoid fever diagnosed between March 2015 and December 2016 at QECH were included in a prospective observational cohort \citep{Feasey:2015} . Age, residential area, human immunodeficiency virus (HIV) status, inpatient versus outpatient treatment, clinical presentation, complications, and deaths were recorded from clinical case records and/or during patient interviews \citep{Feasey:2015}.

\section{The MCET Data}

Most salmonella strains cause gastroenteritis, while some strains, particularly \textit{Salmonella enterica} serotypes Typhi and Paratyphi, are more invasive and typically cause enteric fever. Enteric (typhoid) fever is a more serious infection that poses problems for treatment due to antibacterial resistance (ABR) in many parts of the 

\restoregeometry

world. According to the study on the burden of typhoid fever in low-and middle-income countries (South America, Sub-Saharan Africa and South-East Asia) suggests that 17.8 million (95\% CrI 6.9 to 48.4 million) typhoid fever cases occur annually \citep{Antillon:2017}.

In Malawi, a longitudinal health surveillance study conducted at QECH in Blantyre showed a rapid increase in microbiologically confirmed \textit{S. typhi} infections \citep{Feasey:2015}. For example, 67 typhoid fever cases were confirmed in 2011 followed by 186 cases in 2012 and 843 cases in 2013 and 782 cases in 2014 \citep{Feasey:2015}. In trying to understand the transmission routes of the rapid increase in cases of microbiologically confirmed \textit{S. typhi} infections, the Morbidity, Carriage and genomic Epidemiology of Typhoid (MCET) study was conducted in 2015 \citep{Feasey:2015}. The aim of the study was to investigate whether the typhoid fever cases were caused by a single lineage/haplotype of \textit{S. typhi}, and to describe the full diversity of \textit{S. typhi} in Blantyre.

In the study, patients under the age of 10 diagnosed with culture-confirmed typhoid fever at QECH in Blantyre were recruited in the prospective observational cohort study \citep{Feasey:2015}. The results showed that prior to 2011, typhoid fever cases were being caused by four different \textit{S. typhi} haplotype/lineages (H42,H52,H50 and H55). Typhoid fever cases caused by MDR lineage H58 increased sharply in 2011. By 2013, all typhoid fever cases which were being registered at QECH were caused by H58-haplotype/lineage \citep{Feasey:2015}. Further analysis of the MCET data revealed that there are 7 sub-lineages of the H58 lineage which were causing the typhoid fever outbreak \citep{Gauld:2022}.

\section{Model Specification and Statistical Analysis}

This section discusses specific LGCP models which have been fitted to investigate the spatial and spatio-temporal distribution of the sub-lineages of the H58 lineage of \textit{S. typhi}.

\subsection{Statistical Models}

The MCET study data consists of geo-referenced point pattern data over time. Therefore, a spatio-temporal point process model will be appropriate to model these data. Since the aim of the project is to investigate heterogeneities in process intensity over time and geographical space, the LGCP model, introduced in Chapter 2, provides the flexibility required for this application. The general spatio-temporal and multivariate spatial LGCP equations in  \ref{eqn1} and \ref{eqn2} respectively will be implemented as specified below. The intensity function of the spatio-temporal model will be expressed as

\begin{align}
    \label{eqn3}
    X(s,t) &= Pois \lbrace R(s,t)\rbrace \nonumber \\
    R(s,t) &= C_A \lambda(s,t) exp \lbrace Y(s,t) + S(t)\rbrace \quad \quad where \quad S(t) = \displaystyle \sum_{i=0}^{n} N_{i,k}(t)P_i
\end{align}

In equation \ref{eqn3}, $X(s,t)$ is the number of events in the cell on the computational grid at location $s$ and at time $t$. $R(s,t)$ is the spatially and temporally varying Poisson rate or the intensity of the point process. $C_A$ is the cell area and $\lambda(s,t)$ is the population offset. $Y(s,t)$ is the latent Gaussian process on the computational grid. $S(t)$ is the cubic B-spline to model the temporal trends of the typhoid cases. $\lbrace N_{i,k} \rbrace_{i=1}^n$ are piecewise polynomial basis functions defined using the Cox-de Boor recursion formula below:

\begin{align}
    N_{i,0}(t) = \left\{
    \begin{array}{cl}
        1 & \quad if \quad t_i \leq t < t_{i+1} \\
        0 & \quad otherwise\quad,             \\
    \end{array} \right.
\end{align}

\begin{center}
    $N_{i,k}(t) = \dfrac{t - t_i}{t_{i+k} - t_i} N_{i,k-1}(t) + \dfrac{t_{i+k+1}-t}{t_{i+k+1}-t_{i+1}} N_{i+1,k-1}(t)$
\end{center}

where $k = 0,1,2,3$ is the degree of the spline. $P_i$ are the control points of the B-spline. The three derived vectors of parameter values which affect the shape of the B-spline curve will be represented by $t^{'}, t^{''}$ and $t^{'''}$ in the proceeding sections. The population counts of children aged 10 and below from the 2018 Malawi population census have been used as the population offset for the models. This will help to account for differing population density across the city and to interpret the results as incidence rates instead of counts.\\

Since the MCET study data are marked spatio-temporal point pattern, it is also important to fit a multi-type spatio-temporal model to assess the spatial and temporal distribution of typhoid fever cases caused by specific genomic sub-lineages of the H58 lineage of \textit{S. typhi}. However, due to the time constraints inherent to an MSc project and the limitation of the current \textit{lgcp} R package which cannot fit a multi-type spatio-temporal LGCP model, a multi-type spatial-only LGCP model will be fitted instead. The intensity function for the multi-type spatial LGCP model is defined as
\begin{align}
    \label{eqn4}
    X_m(s) &= Pois \lbrace R_m(s)\rbrace \nonumber \\
    R_m(s) &= C_A \lambda(s) exp \lbrace Y_m(s) + Y_{m+1}(s)\rbrace \quad \quad m =  1,2,3.
\end{align}

In equation \ref{eqn4}, $X_m(s)$ represents the number of events of type $m$ on the computational at location $s$. $R_m(s)$ is the intensity of the LGCP and $C_A$ is the cell area. $\lambda(s)$ is the population offset and $Y_m(s)$ are latent Gaussian processes on the computational grid specific to a particular type. $Y_{m+1}$ is the latent Gaussian process which captures spatial variation common to all types. $m=1,2,3$ because there are 3 types (clade 0, 2 and the combined clade with sub-lineages 1,3,4,5 and 6).\\

Theoretically, LGCP models are formulated on a spatio-temporal continuum. However, actual implementation is done by approximating the study region using a grid of square cells. In this article, 350m was used as grid cell side length. Minimum contrast estimation also known as least-squares approach, a non-parametric method, was used to determine the computational grid size which helped to capture the spatial dependence in the latent Gaussian process \citep{Moller:1998, Davies:2013}.

\subsection{Bayesian Estimation}

Gibbs sampling is one of the Monte Carlo Markov Chain algorithm which can be used to sample from a given multivariate probability distribution. Since it is not possible to sample from this distribution, the Gibbs sampler iteratively draws an instance from the conditional distribution of each variable, conditional on the current values of the other variables for parameter estimation of the multivariate probability distribution. Even though Gibbs sampling is relatively easy to implement, it is not very efficient when tracking many parameters like in spatial and spatio-temporal modelling. The \textit{lgcp} package solved this problem by combining Gibbs sampling with the Metropolis-adjusted Langevin algorithm (MALA). MALA is an adaptive MCMC algorithm which helps the sampler to move towards areas of higher posterior probability \citep{Taylor:2013}.

For valid and reliable statistical inference, the MCMC needs to converge to a stationary distribution. In this thesis, the MCMC for the spatio-temporal LGCP models were run 1,000,000 iterations with 100,000 iterations as burn-in. The MCMC for the multi-type spatial model was run 40,000,000 iterations with 500,000 iterations as burn-in. Pilot runs of 100,000 iterations helped to determine the aforementioned number of iterations. Trace plots were used to assess convergence and to determine the number of iterations.

All statistical analyses were conducted using the R statistical software, version 4.0.3. Firstly, descriptive analyses were done to summarise the data. The LGCP models were fitted using the R package \textit{lgcp}\citep{Taylor:2013, Taylor:2015}. The \textit{lgcp} package uses Bayesian inference to get the joint predictive distribution of the latent Gaussian process where summaries of the predictive distribution for incidence rates and exceedance of a prespecified incidence threshold can be made. The covariance matrix of the latent Gaussian process on the computational grid will be identified by $\sigma$, $\phi$ and $\theta$ parameters. The parameter $\sigma$ scales the log-intensity whereas the parameters $\phi$ and $\theta$ manage the rates at which the correlation function decreases in space and in time respectively \citep{Taylor:2013, Taylor:2015}.

The spatio-temporal analyses in this thesis project assume a separable covariance structure. This means that the correlation between two points in space and time can be decomposed into purely spatial and purely temporal components\ref{2.3.1} \citep{Davies:2013}. An exponential covariance function was used to model the spatial dependency in the Gaussian process. The prior densities for spatio-temporal models are a multivariate normal specified as follows: $\eta = \left\lbrace log \sigma,log \phi, log \theta \right\rbrace  \sim MVN(\bf\mu_{\eta},\bf\Sigma_{\eta}, \bf\Omega_{\eta})$. For the multi-type spatial model will be as follows: the multivariate Gaussian prior for $\beta$ will be $\beta \sim MVN(\bf\mu_{\beta},\bf\Sigma_{\beta})$ and for the multivariate Gaussian prior on the log-scale for the positive parameters $\sigma$ and $\phi$ will be $\eta = \left\lbrace log \sigma,log \phi \right\rbrace  \sim MVN(\bf\mu_{\eta},\bf\Sigma_{\eta})$ \citep{Taylor:2013}.

The four spatio-temporal LGCP models and the multi-type spatial LGCP model were fitted using the \textit{lgcp} package functions \textit{lgcpPredictSpatioTemporalPlusPars} and \textit{lgcpPredictMultitypeSpatialPlusPars} functions respectively. The exceedance and segregation probabilities of the latent Gaussian process were also calculated. Exceedance probability, $P[exp(Y)>k]$, is the probability that the incidence rate exceeds threshold $k$ for each cell on the computational grid. On the other hand, segregation probabilities are used in the multivariate LGCP models to ascertain locations of high probability that a particular location will have an event of a particular type. In this thesis, a sub-lineage was considered dominant in a particular location if the conditional probability of an event of a given type at that location exceeded 0.8 threshold \citep{Taylor:2015}.

\section{Study Outcomes}

The outcome of the thesis are the spatial and spatio-temporal LGCP models which have been fitted with the use of intensity functions specified in section 3.3.1 above. The results of the models will help to describe the spatial and spatio-temporal variations of the distribution of typhoid cases by genomic sub-lineages. The results will also help in understanding if there were spatial and/or temporal interactions of the sub-lineages among typhoid cases.

\section{Ethical Considerations}

The MCET study was approved by the University of Malawi, College of Medicine Research and Ethics Committee (no. P.08/14/1617), the Liverpool School of Tropical Medicine Research Ethics Committee (no. 14.042), and the Lancaster University Faculty of Health and Medicine Ethics Committee (no. FHMREC17014). Informed written consent was sought from adult participants and from the legal guardians of children whose data have been used in this paper.
