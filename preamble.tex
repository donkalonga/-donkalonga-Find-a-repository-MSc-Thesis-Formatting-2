
\setcounter{chapter}{0}
\setcounter{secnumdepth}{0}
\cfoot{\thepage}
\pagenumbering{roman}
\pagenumbering{gobble}

\sectionfont{\centering\MakeUppercase}

\thispagestyle{empty}

\chapter*{Declaration}
%\addcontentsline{toc}{chapter}{Declaration}

In accordance with the regulation of the University of Malawi, I, the undersigned, hereby declare that the work described here is my own original work, except where due reference are made, and has not been submitted for a degree in any university or institution.

\begin{center}
\rule{10cm}{0.2mm}

{\bfseries Full Legal Name}

\vspace*{3cm}

\rule{10cm}{0.2mm}

{\bfseries Signature}

\vspace*{3cm}

\rule{10cm}{0.2mm}

{\bfseries Date}

\end{center}

\newpage

\thispagestyle{empty}

\chapter*{Certificate of Approval}
%\addcontentsline{toc}{chapter}{Certificate of Approval}

The undersigned certify that this thesis represents the student’s own work and effort and has been submitted with my approval

Signature : \rule{5cm}{0.2mm}    Date : \rule{5cm}{0.2mm}

Marc Y. R. Henrion, PhD (Senior Biostatistician)\\
\textbf{Main Supervisor}

\newpage

\thispagestyle{empty}

\chapter*{Dedication}
%\addcontentsline{toc}{chapter}{Dedication}

%\vspace*{\fill}

This thesis is dedicated to my grandmother Maria Kalonga and my aunt Mrs Jane Chinyengo.

%\vspace*{\fill}

\newpage

\thispagestyle{empty}

\chapter*{Acknowledgements}
%\addcontentsline{toc}{chapter}{Acknowledgements}

It has been a long and difficult journey for this thesis and the MSc in general to become a reality. This MSc would not have been possible without the support of many people. I would like to sincerely thank God for His guidance throughout my study.

I would like to thank my supervisor, \emph{Dr Marc Henrion} for the support and encouragement during the development of this thesis. His comments, suggestions and patience played an enormous role in the development of this thesis. Please receive my sincere thanks. I would also like to thanks \emph{Prof. Nick Feasey} for approving the use of the morbidity, carriage and genomic epidemiology of typhoid (MCET) study data for this thesis. Your guidance helped to shape the concept of this thesis.

I would like to thank by boss, \emph{Mr A. Masiye}, for allowing me time off from work to attend classes and all the demands of the MSc. He shouldered most of my responsibilities so that I should have undisrupted studies. I will forever be grateful for his sacrifice. I would also like to thanks \emph{Mr Isaac Kasenjere} for hosting me at his home for the entire period of my studies at the University of Malawi. You are one of the few good men I know. I will forever be grateful.

This acknowledgement would be incomplete without recognizing the invaluable contribution of \emph{Edmond Kachale}, my colleague from my alma mater during my bachelor's degree. I reached out to him in desperation for assistance in typesetting this document in \LaTeX, ensuring strict adherence to the precise guidelines of the University of Malawi. Despite his own academic engagements, he selflessly prioritized my work, surpassing my expectations, and produced this excellent work that you are currently reading. I am deeply grateful for his exceptional skills and efforts.

Special thanks should also go to my wife, \emph{Marie Kalonga}, and my two sons, \emph{Michael Don Kalonga} and \emph{Seth Jared Kalonga} for their moral support during my studies. You have been the reason I did not give up even when the going got tough multiple times.

\newpage

\cfoot{\thepage}
\pagenumbering{roman}
\setcounter{page}{6}

\newgeometry{left=1.5in, right=1in, top=2in, bottom=1in}

\chapter*{ABSTRACT}
\addcontentsline{toc}{chapter}{ABSTRACT}

Typhoid fever is a major cause of morbidity and mortality in low and middle-income settings, with an estimated 10–20 million cases and approximately 200,000 deaths occurring annually. A recent epidemic of typhoid fever in Blantyre saw cases rise from 67 in 2011 to 782 in 2014. The morbidity, carriage and genomic epidemiology of typhoid (MCET) study which was conducted by Malawi-Liverpol Wellcome Trust at Queen Elizabeth Central Hospital between 2015 and 2016 found that 7 genomic sub-lineages of H58 lineage of \textit{S. typhi} were causing the outbreak. The spatial and spatio-temporal distribution of the 7 sub-lineages, which may help to explain their transmission routes, has not yet been described. Log-Gaussian Cox Process (LGCP) models were used for the spatial and spatio-temporal analyses. Specifically, a multi-type spatial LGCP model was used for an overall spatial analysis and four spatio-temporal LGCP models (all cases and stratified by sub-lineage) were fitted for a spatio-temporal analysis of the data. The parameters in the LGCP models include $\sigma$ with median of 2.185 (95\% Credible Interval (CrI) 1.93 to 2.497); the parameter $\phi$ had a median of 940.1 metres (95\% CrI 709 to 1275); and the parameter $\theta$ had a median of 0.075 months (95\% CrI 0.050 to 0.107). $\sigma$ is the standard deviation parameter which scales the log-intensity, whilst the parameters $\phi$ and $\theta$ govern the rates at which the correlation function decreases in space and in time respectively. The long term distribution of the typhoid cases show that the outbreak was at its peak between October and November 2015. The analysis provides evidence for sub-lineage specific spatial distribution of the H58 lineage of \textit{S. typhi} during the recent typhoid outbreak in Blantyre, Malawi.

\restoregeometry

\newpage